\IEEEPARstart{T}{he} rapid evolution of cybersecurity threats coupled with the proliferation of diverse computing platforms has created unprecedented challenges for modern security operations. Organizations today must defend against sophisticated attacks across web applications, desktop systems, and mobile devices, each requiring specialized security considerations and monitoring capabilities. Traditional Security Information and Event Management (SIEM) solutions, while effective in centralized log collection and analysis, often struggle with multi-platform deployment, rely heavily on signature-based detection that fails against novel attacks, and create operational silos that fragment security visibility \cite{splunk2023,ibm2023}.

The cybersecurity landscape has fundamentally transformed in recent years. Attack vectors have evolved from simple signature-based exploits to sophisticated multi-stage campaigns leveraging machine learning for evasion, polymorphic malware that changes signatures dynamically, and zero-day vulnerabilities that bypass traditional detection systems \cite{deeplearning2023}. Simultaneously, the attack surface has expanded dramatically with the proliferation of web applications, mobile devices, Internet of Things (IoT) endpoints, and cloud infrastructure, each presenting unique security challenges and requiring specialized monitoring approaches \cite{kumar2023}.

Current security operations face several critical limitations. First, platform fragmentation forces organizations to deploy separate security solutions for web, desktop, and mobile environments, creating inconsistent policies, fragmented visibility, and operational overhead. Second, signature-based detection approaches, while computationally efficient, fail to identify novel attack patterns and zero-day exploits, resulting in high false negative rates for emerging threats. Third, centralized log storage in traditional databases remains vulnerable to tampering, undermining forensic analysis and compliance requirements. Fourth, manual deployment and configuration of security infrastructure leads to human errors, configuration drift, and inconsistent security postures across environments \cite{chen2023,zhang2023}.

\subsection{Motivation}

The motivation for developing a unified, multi-platform security operations framework with hybrid threat detection stems from several converging factors in modern cybersecurity operations:

\textbf{Multi-Platform Security Requirements:} Modern organizations operate in heterogeneous environments where users access systems through web browsers, native desktop applications, and mobile devices. Each platform presents unique security considerations: web applications face Cross-Site Scripting (XSS) and SQL injection attacks; desktop systems encounter malware and privilege escalation attempts; mobile devices suffer from app-based attacks and insecure data storage. Existing SIEM solutions primarily target server-side and network-level monitoring, providing limited visibility into client-side attacks and platform-specific threats. A unified platform that maintains consistent security policies while accommodating platform-specific requirements addresses this critical gap.

\textbf{Limitations of Signature-Based Detection:} Traditional pattern-matching approaches rely on predefined attack signatures, making them ineffective against zero-day exploits, polymorphic malware, and sophisticated evasion techniques. The average time to detect a breach remains 287 days according to recent industry reports \cite{verizon2023}, during which attackers can exfiltrate sensitive data, establish persistence, and cause significant damage. Machine learning approaches offer promise for detecting novel attack patterns through behavioral analysis and anomaly detection, but integration with production security systems remains challenging due to false positive rates, training data requirements, and operational complexity.

\textbf{Compliance and Audit Requirements:} Regulatory frameworks including GDPR, HIPAA, SOX, and PCI-DSS mandate comprehensive audit trails, data integrity verification, and tamper-proof logging for security events. Traditional database storage, while performant, lacks inherent immutability guarantees and remains vulnerable to sophisticated attackers who compromise logging infrastructure to cover their tracks. Blockchain technology offers potential for immutable audit trails, but practical integration with high-throughput security operations remains an open research challenge.

\textbf{Deployment Complexity:} Enterprise security infrastructure deployment traditionally requires extensive manual configuration, custom integration code, and specialized expertise for setup and maintenance. Infrastructure as Code (IaC) approaches using tools like Terraform and Ansible promise automated deployment, configuration consistency, and reduced time-to-production, but integration with security-specific requirements and multi-platform considerations requires careful architectural design.

\textbf{Educational and Research Value:} Beyond operational requirements, there exists significant need for educational platforms that enable hands-on learning about security threats, detection mechanisms, and defensive strategies. Interactive security laboratories that allow safe experimentation with attack patterns and detection systems provide valuable learning experiences for students, security professionals, and researchers.

\subsection{Research Objectives and Contributions}

This research addresses the aforementioned challenges through the design, implementation, and evaluation of USOD (Unified Security Operations Dashboard), a comprehensive security operations platform. The primary research objectives include:

\begin{enumerate}
\item \textbf{Design and implement} a unified multi-platform security architecture that provides consistent security operations across web, desktop, and mobile environments while accommodating platform-specific requirements and constraints.

\item \textbf{Develop and validate} a hybrid threat detection approach combining pattern-based application-layer security with machine learning-based network-layer analysis, evaluating effectiveness, performance characteristics, and operational trade-offs.

\item \textbf{Integrate and evaluate} modern technologies including Server-Sent Events for real-time updates, microservices architecture for scalability, and FastAPI-based ML service integration with traditional web backends.

\item \textbf{Design architectural frameworks} for blockchain-based immutable logging and cloud automation infrastructure, providing detailed specifications and implementation roadmaps for future enhancement.

\item \textbf{Provide comprehensive performance evaluation} including validated metrics, honest assessment of limitations, and clear distinction between implemented capabilities and future work.

\item \textbf{Develop educational security laboratory} enabling hands-on learning about attack patterns, detection mechanisms, and security operations.
\end{enumerate}

The primary contributions of this work include:

\begin{itemize}
\item \textbf{Multi-Platform Unified Architecture:} Complete implementation of security operations dashboard across web (Next.js 15.5.2 with React 19.1.0), desktop (Electron 38.2.2 with React 18.2.0), and mobile (React Native 0.81.4 with Expo 54.0.13) platforms, demonstrating practical cross-platform development patterns and shared backend integration.

\item \textbf{Hybrid Threat Detection:} Working implementation combining pattern-based detection for 12 application-layer attack types with ML-based network analysis using Random Forest and Isolation Forest models, including honest evaluation of strengths, limitations, and dataset dependencies.

\item \textbf{Real-Time Integration Architecture:} Production implementation of Python FastAPI ML service integrated with Node.js Express 5 backend via HTTP webhooks and Server-Sent Events, achieving sub-100ms end-to-end latency for threat distribution.

\item \textbf{Comprehensive Logging System:} Implementation of 30 distinct event types capturing security events across application and network layers, with MongoDB storage optimized for retrieval and analysis.

\item \textbf{Educational Security Laboratory:} Interactive testing environment supporting 12 attack pattern types with real-time detection feedback, providing hands-on learning capabilities.

\item \textbf{Validated Performance Metrics:} Extensive evaluation providing honest assessment of system capabilities, clearly distinguishing validated metrics (99.97\% accuracy on CICIDS2017, 34.51ms ML inference time, sub-200ms response times) from estimated targets and future work.

\item \textbf{Architectural Designs for Future Enhancement:} Complete documentation and specifications for blockchain integration (Hyperledger Fabric) and cloud automation (Terraform/Ansible), providing roadmap for enterprise deployment.

\item \textbf{Honest Assessment of Limitations:} Transparent acknowledgment of current limitations including ML model dataset scope (CICIDS2017 from 2017 showing 2-19\% confidence on modern malware), unimplemented blockchain and cloud automation components, and areas requiring future research and development.
\end{itemize}

\subsection{Paper Organization}

The remainder of this paper is organized as follows: Section II provides background on security operations, multi-platform development challenges, and machine learning for threat detection. Section III surveys related work in SIEM systems, multi-platform security solutions, AI-enhanced detection, and blockchain applications in security. Section IV presents the overall system architecture including multi-tier design, component integration, and extensibility framework. Section V details multi-platform implementation across web, desktop, and mobile platforms with specific technical considerations. Section VI describes security detection mechanisms including pattern-based application-layer detection and automated response systems. Section VII presents the AI-enhanced network threat detection implementation including ML pipeline, model architecture, and integration with existing security infrastructure. Section VIII covers data management and comprehensive logging system design. Section IX describes real-time communication infrastructure using Server-Sent Events. Section X discusses user interface design and cross-platform user experience considerations. Section XI provides comprehensive performance evaluation with validated metrics and honest assessment of limitations. Section XII analyzes security aspects of the system itself. Section XIII discusses implications, trade-offs, and design decisions. Section XIV presents lessons learned and best practices from development and deployment. Section XV outlines future work and research directions. Section XVI concludes the paper. Appendices provide detailed system specifications, extended performance metrics, and code samples.

