This appendix provides comprehensive deployment configurations, infrastructure specifications, and operational procedures for deploying USOD in various environments including development, educational, and production scenarios.

\subsection{System Requirements}

\subsubsection{Backend Server Requirements}

\textbf{Minimum Specifications (Development/Educational)}:
\begin{itemize}
    \item CPU: 2 cores, 2.0 GHz+
    \item RAM: 4 GB
    \item Storage: 20 GB SSD
    \item Network: 10 Mbps+
    \item OS: Windows 10/11, macOS 10.15+, Ubuntu 20.04+
\end{itemize}

\textbf{Recommended Specifications (Production)}:
\begin{itemize}
    \item CPU: 8 cores, 3.0 GHz+
    \item RAM: 32 GB
    \item Storage: 500 GB NVMe SSD
    \item Network: 1 Gbps+
    \item OS: Ubuntu 22.04 LTS, RHEL 8+
\end{itemize}

\subsubsection{AI Service Requirements}

\textbf{Python AI Service}:
\begin{itemize}
    \item CPU: 4+ cores (8+ recommended for real-time processing)
    \item RAM: 8 GB minimum (16 GB recommended)
    \item Storage: 50 GB (for models and packet captures)
    \item GPU: Optional, not utilized by current ML models
    \item Network: Low-latency connection to backend (<5ms preferred)
\end{itemize}

\subsubsection{Database Requirements}

\textbf{MongoDB}:
\begin{itemize}
    \item Version: MongoDB 5.0+ (6.0+ recommended)
    \item RAM: 8 GB minimum (more for larger datasets)
    \item Storage: 100 GB+ SSD (growth rate ~1 GB per 100K events)
    \item IOPS: 3000+ for production workloads
    \item Replication: 3-node replica set for production
\end{itemize}

\subsection{Software Dependencies}

\subsubsection{Node.js Backend Dependencies}

\begin{lstlisting}[language=JSON, caption=package.json Dependencies, basicstyle=\footnotesize\ttfamily, breaklines=true]
{
  "dependencies": {
    "express": "^5.1.0",
    "mongoose": "^8.0.0",
    "jsonwebtoken": "^9.0.2",
    "bcrypt": "^5.1.1",
    "cors": "^2.8.5",
    "helmet": "^7.1.0",
    "express-rate-limit": "^7.1.5",
    "express-mongo-sanitize": "^2.2.0",
    "express-validator": "^7.0.1",
    "dotenv": "^16.3.1",
    "axios": "^1.6.2"
  },
  "devDependencies": {
    "nodemon": "^3.0.2",
    "jest": "^29.7.0",
    "supertest": "^6.3.3",
    "eslint": "^8.56.0"
  }
}
\end{lstlisting}

\textbf{Node.js Version}: 18.x LTS or 20.x LTS recommended

\subsubsection{Python AI Service Dependencies}

\begin{lstlisting}[caption=requirements.txt, basicstyle=\footnotesize\ttfamily, breaklines=true]
fastapi==0.104.1
uvicorn[standard]==0.24.0
pandas==2.1.3
numpy==1.26.2
scikit-learn==1.3.2
scapy==2.5.0
joblib==1.3.2
python-multipart==0.0.6
pydantic==2.5.2
matplotlib==3.8.2
seaborn==0.13.0
requests==2.31.0
\end{lstlisting}

\textbf{Python Version}: Python 3.9, 3.10, or 3.11 (3.10 recommended)

\subsubsection{Frontend Dependencies}

\textbf{Web (Next.js)}:
\begin{lstlisting}[language=JSON, caption=Web Platform Dependencies, basicstyle=\footnotesize\ttfamily, breaklines=true]
{
  "dependencies": {
    "next": "15.5.2",
    "react": "19.1.0",
    "react-dom": "19.1.0",
    "tailwindcss": "^4.0.0"
  }
}
\end{lstlisting}

\textbf{Desktop (Electron)}:
\begin{lstlisting}[language=JSON, caption=Desktop Platform Dependencies, basicstyle=\footnotesize\ttfamily, breaklines=true]
{
  "dependencies": {
    "electron": "38.2.2",
    "react": "18.2.0",
    "react-dom": "18.2.0",
    "react-router-dom": "6.8.0"
  }
}
\end{lstlisting}

\textbf{Mobile (React Native/Expo)}:
\begin{lstlisting}[language=JSON, caption=Mobile Platform Dependencies, basicstyle=\footnotesize\ttfamily, breaklines=true]
{
  "dependencies": {
    "expo": "~54.0.13",
    "react": "19.1.0",
    "react-native": "0.81.4",
    "@react-navigation/native": "^7.1.18",
    "@react-native-async-storage/async-storage": "2.1.0"
  }
}
\end{lstlisting}

\subsection{Environment Configuration}

\subsubsection{Backend Environment Variables}

\begin{lstlisting}[caption=.env Configuration, basicstyle=\footnotesize\ttfamily, breaklines=true]
# Server Configuration
NODE_ENV=production
PORT=5000
HOST=0.0.0.0

# MongoDB Configuration
MONGODB_URI=mongodb://localhost:27017/usod_security
# For replica set:
# MONGODB_URI=mongodb://mongo1:27017,mongo2:27017,mongo3:27017/usod_security?replicaSet=rs0

# JWT Configuration
JWT_SECRET=<generate-strong-256-bit-secret>
JWT_EXPIRES_IN=24h
JWT_REFRESH_EXPIRES_IN=7d

# Security Configuration
BCRYPT_ROUNDS=12
RATE_LIMIT_WINDOW=15
RATE_LIMIT_MAX_REQUESTS=100
CORS_ORIGIN=http://localhost:3000,http://localhost:8080

# Python AI Service
PYTHON_AI_SERVICE_URL=http://localhost:8000
AI_SERVICE_API_KEY=<generate-api-key>

# Logging
LOG_LEVEL=info
LOG_TO_FILE=true
LOG_FILE_PATH=./logs/app.log

# Backup
BACKUP_ENABLED=true
BACKUP_INTERVAL=86400000
BACKUP_PATH=./backups
\end{lstlisting}

\subsubsection{Python AI Service Configuration}

\begin{lstlisting}[caption=.env for Python Service, basicstyle=\footnotesize\ttfamily, breaklines=true]
# FastAPI Configuration
HOST=0.0.0.0
PORT=8000
WORKERS=4

# Node.js Backend
BACKEND_URL=http://localhost:5000
WEBHOOK_ENDPOINT=/api/network/webhook
API_KEY=<same-as-backend-ai-service-api-key>

# Model Configuration
MODEL_PATH=./data/processed
RANDOM_FOREST_MODEL=random_forest_model.pkl
ISOLATION_FOREST_MODEL=isolation_forest_model.pkl
SCALER_MODEL=scaler.pkl
LABEL_ENCODER=label_encoder.pkl

# Detection Configuration
CONFIDENCE_THRESHOLD=0.7
ANOMALY_THRESHOLD=-0.5
BATCH_SIZE=100

# Packet Capture
PCAP_PATH=./data/captures
MAX_PCAP_SIZE_MB=1000
CAPTURE_TIMEOUT=300
\end{lstlisting}

\subsection{Docker Deployment}

\subsubsection{Docker Compose Configuration}

\begin{lstlisting}[language=yaml, caption=docker-compose.yml, basicstyle=\footnotesize\ttfamily, breaklines=true]
version: '3.8'

services:
  mongodb:
    image: mongo:6.0
    container_name: usod-mongodb
    restart: unless-stopped
    environment:
      MONGO_INITDB_DATABASE: usod_security
    volumes:
      - mongodb-data:/data/db
      - ./mongo-init:/docker-entrypoint-initdb.d
    ports:
      - "27017:27017"
    networks:
      - usod-network
    command: mongod --bind_ip_all

  backend:
    build:
      context: ./backend
      dockerfile: Dockerfile
    container_name: usod-backend
    restart: unless-stopped
    depends_on:
      - mongodb
    environment:
      - NODE_ENV=production
      - MONGODB_URI=mongodb://mongodb:27017/usod_security
      - PORT=5000
    env_file:
      - ./backend/.env
    ports:
      - "5000:5000"
    volumes:
      - ./backend/uploads:/app/uploads
      - ./backend/backups:/app/backups
      - ./backend/logs:/app/logs
    networks:
      - usod-network
    healthcheck:
      test: ["CMD", "curl", "-f", "http://localhost:5000/health"]
      interval: 30s
      timeout: 10s
      retries: 3

  ai-service:
    build:
      context: ./ai
      dockerfile: Dockerfile
    container_name: usod-ai-service
    restart: unless-stopped
    depends_on:
      - backend
    environment:
      - HOST=0.0.0.0
      - PORT=8000
      - BACKEND_URL=http://backend:5000
    env_file:
      - ./ai/.env
    ports:
      - "8000:8000"
    volumes:
      - ./ai/data:/app/data
      - ./ai/models:/app/models
    networks:
      - usod-network
    # Privileged mode required for packet capture
    # Remove for demo mode
    privileged: true
    cap_add:
      - NET_ADMIN
      - NET_RAW

  web:
    build:
      context: ./frontend
      dockerfile: Dockerfile
    container_name: usod-web
    restart: unless-stopped
    depends_on:
      - backend
    environment:
      - NEXT_PUBLIC_API_URL=http://localhost:5000
      - NEXT_PUBLIC_WS_URL=ws://localhost:5000
    ports:
      - "3000:3000"
    networks:
      - usod-network

volumes:
  mongodb-data:
    driver: local

networks:
  usod-network:
    driver: bridge
\end{lstlisting}

\subsubsection{Backend Dockerfile}

\begin{lstlisting}[caption=backend/Dockerfile, basicstyle=\footnotesize\ttfamily, breaklines=true]
FROM node:20-alpine

WORKDIR /app

# Copy package files
COPY package*.json ./

# Install dependencies
RUN npm ci --only=production

# Copy application code
COPY . .

# Create necessary directories
RUN mkdir -p uploads backups logs

# Expose port
EXPOSE 5000

# Health check
HEALTHCHECK --interval=30s --timeout=10s --start-period=5s \
  CMD node healthcheck.js

# Start application
CMD ["node", "src/server.js"]
\end{lstlisting}

\subsubsection{AI Service Dockerfile}

\begin{lstlisting}[caption=ai/Dockerfile, basicstyle=\footnotesize\ttfamily, breaklines=true]
FROM python:3.10-slim

WORKDIR /app

# Install system dependencies
RUN apt-get update && apt-get install -y \
    gcc \
    g++ \
    libpcap-dev \
    tcpdump \
    && rm -rf /var/lib/apt/lists/*

# Copy requirements
COPY requirements.txt .

# Install Python dependencies
RUN pip install --no-cache-dir -r requirements.txt

# Copy application code
COPY . .

# Create directories
RUN mkdir -p data/processed data/captures models

# Expose port
EXPOSE 8000

# Start FastAPI with uvicorn
CMD ["uvicorn", "main:app", "--host", "0.0.0.0", \
     "--port", "8000", "--workers", "4"]
\end{lstlisting}

\subsection{MongoDB Configuration}

\subsubsection{Replica Set Setup (Production)}

\begin{lstlisting}[caption=MongoDB Replica Set Initialization, basicstyle=\footnotesize\ttfamily, breaklines=true]
# Connect to primary node
mongosh mongodb://localhost:27017

# Initialize replica set
rs.initiate({
  _id: "rs0",
  members: [
    { _id: 0, host: "mongo1:27017", priority: 2 },
    { _id: 1, host: "mongo2:27017", priority: 1 },
    { _id: 2, host: "mongo3:27017", priority: 1 }
  ]
})

# Verify status
rs.status()
\end{lstlisting}

\subsubsection{MongoDB Indexes}

Create indexes for optimal query performance:

\begin{lstlisting}[language=JavaScript, caption=MongoDB Index Creation, basicstyle=\footnotesize\ttfamily, breaklines=true]
// Connect to database
use usod_security

// SecurityLog collection indexes
db.securitylogs.createIndex({ userId: 1 })
db.securitylogs.createIndex({ action: 1 })
db.securitylogs.createIndex({ status: 1 })
db.securitylogs.createIndex({ platform: 1 })
db.securitylogs.createIndex({ timestamp: -1 })
db.securitylogs.createIndex({ platform: 1, timestamp: -1 })
db.securitylogs.createIndex({ userId: 1, platform: 1 })
db.securitylogs.createIndex({ ipAddress: 1 })
db.securitylogs.createIndex({ "details.eventType": 1 })

// Users collection indexes
db.users.createIndex({ username: 1 }, { unique: true })
db.users.createIndex({ email: 1 }, { unique: true })
db.users.createIndex({ role: 1 })

// Verify indexes
db.securitylogs.getIndexes()
db.users.getIndexes()
\end{lstlisting}

\subsection{Nginx Reverse Proxy Configuration}

For production deployments, Nginx provides load balancing, SSL termination, and rate limiting:

\begin{lstlisting}[caption=nginx.conf, basicstyle=\footnotesize\ttfamily, breaklines=true]
upstream backend {
    least_conn;
    server backend1:5000 max_fails=3 fail_timeout=30s;
    server backend2:5000 max_fails=3 fail_timeout=30s;
    server backend3:5000 max_fails=3 fail_timeout=30s;
}

upstream ai_service {
    server ai-service1:8000;
    server ai-service2:8000;
}

# Rate limiting
limit_req_zone $binary_remote_addr zone=api_limit:10m rate=10r/s;
limit_req_zone $binary_remote_addr zone=auth_limit:10m rate=5r/m;

server {
    listen 80;
    listen [::]:80;
    server_name usod.example.com;
    
    # Redirect to HTTPS
    return 301 https://$server_name$request_uri;
}

server {
    listen 443 ssl http2;
    listen [::]:443 ssl http2;
    server_name usod.example.com;

    # SSL Configuration
    ssl_certificate /etc/letsencrypt/live/usod.example.com/fullchain.pem;
    ssl_certificate_key /etc/letsencrypt/live/usod.example.com/privkey.pem;
    ssl_protocols TLSv1.2 TLSv1.3;
    ssl_ciphers HIGH:!aNULL:!MD5;
    ssl_prefer_server_ciphers on;

    # Security Headers
    add_header Strict-Transport-Security "max-age=31536000; includeSubDomains" always;
    add_header X-Frame-Options "SAMEORIGIN" always;
    add_header X-Content-Type-Options "nosniff" always;
    add_header X-XSS-Protection "1; mode=block" always;

    # API endpoints
    location /api/ {
        limit_req zone=api_limit burst=20 nodelay;
        
        proxy_pass http://backend;
        proxy_http_version 1.1;
        proxy_set_header Upgrade $http_upgrade;
        proxy_set_header Connection 'upgrade';
        proxy_set_header Host $host;
        proxy_set_header X-Real-IP $remote_addr;
        proxy_set_header X-Forwarded-For $proxy_add_x_forwarded_for;
        proxy_set_header X-Forwarded-Proto $scheme;
        proxy_cache_bypass $http_upgrade;
        
        # Timeouts
        proxy_connect_timeout 60s;
        proxy_send_timeout 60s;
        proxy_read_timeout 60s;
    }

    # Authentication endpoints (stricter rate limit)
    location /api/auth/ {
        limit_req zone=auth_limit burst=5 nodelay;
        
        proxy_pass http://backend;
        proxy_set_header Host $host;
        proxy_set_header X-Real-IP $remote_addr;
    }

    # SSE endpoints (longer timeout)
    location /api/network/stream {
        proxy_pass http://backend;
        proxy_http_version 1.1;
        proxy_set_header Connection '';
        proxy_buffering off;
        proxy_cache off;
        proxy_read_timeout 86400s;
        chunked_transfer_encoding on;
    }

    # AI Service
    location /ai/ {
        proxy_pass http://ai_service/;
        proxy_set_header Host $host;
        proxy_set_header X-Real-IP $remote_addr;
    }

    # Static files
    location / {
        root /var/www/usod/frontend;
        try_files $uri $uri/ /index.html;
        
        # Caching
        expires 1y;
        add_header Cache-Control "public, immutable";
    }

    # Gzip compression
    gzip on;
    gzip_vary on;
    gzip_min_length 1000;
    gzip_types text/plain text/css text/xml text/javascript 
               application/x-javascript application/xml+rss 
               application/json application/javascript;
}
\end{lstlisting}

\subsection{Systemd Service Files}

For Linux deployments without Docker:

\subsubsection{Backend Service}

\begin{lstlisting}[caption=/etc/systemd/system/usod-backend.service, basicstyle=\footnotesize\ttfamily, breaklines=true]
[Unit]
Description=USOD Backend Service
After=network.target mongodb.service
Requires=mongodb.service

[Service]
Type=simple
User=usod
Group=usod
WorkingDirectory=/opt/usod/backend
Environment=NODE_ENV=production
EnvironmentFile=/opt/usod/backend/.env
ExecStart=/usr/bin/node src/server.js
Restart=on-failure
RestartSec=10
StandardOutput=journal
StandardError=journal

# Security hardening
NoNewPrivileges=true
PrivateTmp=true
ProtectSystem=strict
ProtectHome=true
ReadWritePaths=/opt/usod/backend/uploads /opt/usod/backend/backups

[Install]
WantedBy=multi-user.target
\end{lstlisting}

\subsubsection{AI Service}

\begin{lstlisting}[caption=/etc/systemd/system/usod-ai.service, basicstyle=\footnotesize\ttfamily, breaklines=true]
[Unit]
Description=USOD AI Service
After=network.target usod-backend.service
Requires=usod-backend.service

[Service]
Type=simple
User=root
Group=root
WorkingDirectory=/opt/usod/ai
EnvironmentFile=/opt/usod/ai/.env
ExecStart=/usr/bin/python3 -m uvicorn main:app \
          --host 0.0.0.0 --port 8000 --workers 4
Restart=on-failure
RestartSec=10

# Packet capture requires root or CAP_NET_RAW
AmbientCapabilities=CAP_NET_RAW CAP_NET_ADMIN

[Install]
WantedBy=multi-user.target
\end{lstlisting}

\subsection{Deployment Procedures}

\subsubsection{Initial Deployment}

\begin{lstlisting}[language=bash, caption=Deployment Script, basicstyle=\footnotesize\ttfamily, breaklines=true]
#!/bin/bash
# deploy.sh - USOD deployment script

set -e

echo "Starting USOD deployment..."

# 1. Clone repository
git clone https://github.com/your-org/usod.git /opt/usod
cd /opt/usod

# 2. Install dependencies
echo "Installing backend dependencies..."
cd backend && npm ci --only=production && cd ..

echo "Installing AI service dependencies..."
cd ai && pip install -r requirements.txt && cd ..

echo "Installing frontend dependencies..."
cd frontend && npm ci --only=production && npm run build && cd ..

# 3. Configure environment
echo "Configuring environment..."
cp backend/.env.example backend/.env
cp ai/.env.example ai/.env

# 4. Generate secrets
JWT_SECRET=$(openssl rand -hex 32)
AI_API_KEY=$(openssl rand -hex 32)

sed -i "s/JWT_SECRET=.*/JWT_SECRET=$JWT_SECRET/" backend/.env
sed -i "s/AI_SERVICE_API_KEY=.*/AI_SERVICE_API_KEY=$AI_API_KEY/" backend/.env
sed -i "s/API_KEY=.*/API_KEY=$AI_API_KEY/" ai/.env

# 5. Setup MongoDB
echo "Setting up MongoDB..."
mongo < scripts/init-mongodb.js

# 6. Create systemd services
echo "Installing systemd services..."
sudo cp scripts/usod-backend.service /etc/systemd/system/
sudo cp scripts/usod-ai.service /etc/systemd/system/
sudo systemctl daemon-reload

# 7. Start services
echo "Starting services..."
sudo systemctl start mongodb
sudo systemctl start usod-backend
sudo systemctl start usod-ai

# 8. Enable auto-start
sudo systemctl enable mongodb usod-backend usod-ai

echo "Deployment complete!"
echo "Backend: http://localhost:5000"
echo "AI Service: http://localhost:8000"
echo "Frontend: Build frontend and serve with nginx"
\end{lstlisting}

\subsubsection{Update/Upgrade Procedure}

\begin{lstlisting}[language=bash, caption=Update Script, basicstyle=\footnotesize\ttfamily, breaklines=true]
#!/bin/bash
# update.sh - USOD update script

set -e

echo "Starting USOD update..."

# 1. Backup
echo "Creating backup..."
./scripts/backup.sh

# 2. Stop services
sudo systemctl stop usod-backend usod-ai

# 3. Pull updates
git pull origin main

# 4. Update dependencies
cd backend && npm ci --only=production && cd ..
cd ai && pip install -r requirements.txt && cd ..

# 5. Run database migrations if any
# mongo < scripts/migrations/$(date +%Y%m%d).js

# 6. Restart services
sudo systemctl start usod-backend usod-ai

# 7. Verify health
sleep 10
curl -f http://localhost:5000/health || exit 1
curl -f http://localhost:8000/health || exit 1

echo "Update complete!"
\end{lstlisting}

These deployment configurations provide comprehensive guidance for deploying USOD across development, educational, and production environments with appropriate security hardening, scalability, and reliability measures.

