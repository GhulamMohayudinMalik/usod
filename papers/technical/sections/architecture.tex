USOD employs a sophisticated multi-layered architecture designed to provide unified security operations across diverse platforms while maintaining high performance, scalability, and extensibility. The system follows microservices principles with event-driven communication patterns, ensuring modularity and maintainability.

\subsection{Overall System Design}

The USOD architecture is built on a microservices foundation that separates concerns across multiple specialized components. The core system consists of a centralized backend API server, a unified security detection engine, and multiple client applications that provide platform-specific user interfaces. This design enables independent scaling of components and facilitates maintenance and updates without system-wide downtime.

The system implements an event-driven architecture using an internal event bus that enables real-time communication between components. This approach ensures loose coupling between services while maintaining high responsiveness for security-critical operations. The event bus supports both synchronous and asynchronous communication patterns, allowing for immediate threat response while enabling background processing for non-critical operations.

\begin{figure}[h]
\centering
\includegraphics[width=0.8\columnwidth]{figures/system-architecture.png}
\caption{USOD Overall System Architecture}
\label{fig:system-architecture}
\end{figure}

\subsection{Multi-Platform Architecture}

USOD provides unified security operations across three distinct platforms, each optimized for its specific environment while maintaining consistent functionality and user experience.

\textbf{Web Platform}: The web application is built using Next.js 15.5.2 with React 19.1.0 and Turbopack for optimized builds, providing server-side rendering capabilities and optimal performance. The application communicates with the Node.js (Express 5.1.0) backend through RESTful APIs and Server-Sent Events (SSE) for real-time updates. Tailwind CSS 4 provides responsive design ensuring consistent functionality across desktop and mobile browsers with sub-200ms average response times.

\textbf{Desktop Platform}: The desktop application leverages Electron 38.2.2 with React 18.2.0 to provide native desktop functionality while maintaining code reuse with the web platform through React Router 6.8.0. The application includes custom focus handling for Electron input optimization, native notifications, and glass-morphism design with dark theme. Full backend integration provides real-time data synchronization across all platforms.

\textbf{Mobile Platform}: The mobile application is developed using React Native 0.81.4 with Expo 54.0.13 and React 19.1.0, ensuring cross-platform compatibility between iOS and Android devices. The application uses React Navigation 7.1.18 for navigation management, AsyncStorage 2.1.0 for local data persistence, and provides touch-optimized interfaces with real backend API integration for all security operations.

All three platforms share a common backend API and security engine, ensuring consistent security policies and threat detection across all environments. The unified backend provides a single source of truth for security data and enables centralized management of security operations.

\subsection{Security Detection Engine}

The security detection engine forms the core of USOD's threat detection capabilities, implementing a multi-layered approach to identify and respond to security threats in real-time.

\textbf{Pattern-Based Detection}: The engine implements comprehensive pattern matching for 12+ attack types including SQL injection, XSS, CSRF, LDAP injection, NoSQL injection, command injection, path traversal, SSRF, XXE, and information disclosure attacks. Each attack type is defined using regular expressions and behavioral patterns that are continuously updated based on emerging threats.

\textbf{Real-Time Processing}: Security detection operates in real-time with sub-200ms response times for threat identification and response. The engine processes incoming requests through a multi-stage pipeline that includes input validation, pattern matching, behavioral analysis, and response generation.

\textbf{Event Bus System}: The internal event bus enables immediate communication between detection components and response mechanisms. When a threat is detected, the event bus triggers immediate IP blocking, logging, and notification processes without requiring database queries or external service calls.

\textbf{IP Management System}: The system maintains dynamic IP blocking capabilities with configurable thresholds and timeouts. Suspicious IPs are tracked using sliding window algorithms, and automatic unblocking occurs after specified time periods or manual intervention by administrators.

\subsection{Data Flow Architecture}

The data flow architecture ensures efficient processing of security events while maintaining data integrity and enabling comprehensive audit trails.

\textbf{Log Ingestion Pipeline}: Security events are ingested through multiple channels including direct API calls, file uploads, and real-time streaming. The ingestion pipeline validates data formats, enriches events with metadata, and routes events to appropriate processing components.

\textbf{Real-Time Event Processing}: Events are processed through a streaming pipeline that performs immediate threat detection, data enrichment, and response generation. The pipeline supports parallel processing to handle high-volume event streams while maintaining low latency.

\textbf{Data Storage and Retrieval}: Security events are stored in MongoDB with optimized indexing for fast retrieval and analysis. The system implements data retention policies and automated archival processes to manage storage requirements while maintaining accessibility for forensic analysis.

\textbf{Blockchain Integration}: The system integrates with Hyperledger Fabric blockchain for immutable audit trails and tamper-proof logging. The ThreatLogContract chaincode provides 10 functions for threat log management including creation, retrieval, filtering, and cryptographic verification. Events are logged to both MongoDB for fast querying and blockchain for immutability, ensuring dual-layer data persistence with comprehensive audit capabilities.

\subsection{Extensibility Framework}

USOD is designed with extensibility as a core principle, enabling easy integration of new security features and platform support.

\textbf{Plugin Architecture}: The system supports a plugin-based architecture that allows for dynamic loading of security detection modules, response handlers, and integration adapters. Plugins can be developed independently and deployed without system restarts.

\textbf{API-Based Integration}: All system functionality is exposed through well-defined REST APIs and WebSocket interfaces, enabling third-party integrations and custom client applications. The API design follows OpenAPI specifications for automatic documentation and client generation.

\textbf{Modular Security Patterns}: Security detection patterns are implemented as modular components that can be easily updated, extended, or replaced. New attack patterns can be added through configuration files without code modifications.

\textbf{Future Enhancement Support}: The architecture includes hooks and interfaces for planned enhancements including AI-powered threat detection, advanced analytics, and additional platform support. The event-driven design ensures that new components can be integrated without disrupting existing functionality.

\subsection{Security Considerations}

Security is embedded throughout the USOD architecture, with multiple layers of protection ensuring the integrity and confidentiality of security operations.

\textbf{Authentication and Authorization}: The system implements JWT-based authentication with role-based access control (RBAC). Multi-factor authentication is supported for administrative accounts, and session management includes automatic timeout and refresh mechanisms.

\textbf{Data Encryption}: All data transmission uses TLS 1.3 encryption, and sensitive data is encrypted at rest using AES-256. Database connections are secured with encrypted connections, and API keys are stored using secure hashing algorithms.

\textbf{Secure Communication}: Inter-service communication uses encrypted channels with certificate-based authentication. The event bus implements message signing to ensure data integrity and prevent tampering.

\textbf{Access Control}: Fine-grained access control is implemented at the API level, with permissions based on user roles and resource ownership. Administrative functions require elevated privileges and are logged for audit purposes.

\textbf{Audit Trails}: All security operations are logged with comprehensive audit trails including user actions, system events, and security decisions. Audit logs are tamper-proof and include cryptographic signatures for integrity verification.

