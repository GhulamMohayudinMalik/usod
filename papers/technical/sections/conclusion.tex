This paper has presented USOD (Unified Security Operations Dashboard), a comprehensive security operations platform that addresses critical challenges in modern cybersecurity through multi-platform integration and hybrid security detection approaches combining pattern-based and ML-based threat identification. The system demonstrates practical implementation of unified security operations across web (Next.js 15/React 19), desktop (Electron 38), and mobile (React Native/Expo 54) platforms.

\subsection{Summary of Contributions}

The primary contributions of this work include: (1) a unified multi-platform security operations framework with consistent interfaces and backend integration across web, desktop, and mobile environments, demonstrating practical cross-platform development; (2) hybrid threat detection combining application-layer pattern-based detection (12 attack types) with network-layer ML-based detection using Random Forest (99.97\% accuracy on CICIDS2017) and Isolation Forest (87.33\% accuracy) models; (3) production-ready Python FastAPI service integrated with Node.js backend via webhooks and Server-Sent Events for real-time threat streaming; (4) comprehensive logging system capturing 30 event types across application and network layers with MongoDB storage; and (5) complete architectural design and documentation for future blockchain (Hyperledger Fabric) and cloud automation (Terraform/Ansible) integration, demonstrating extensibility for enterprise deployment.

\subsection{Key Achievements}

The implementation demonstrates several key achievements validated through development and testing. The multi-platform architecture successfully provides unified security operations with sub-200ms average response times across web, desktop, and mobile platforms. Pattern-based security detection successfully identifies 12 attack pattern types with automatic IP blocking and comprehensive event logging. ML-based network threat detection achieves 99.97\% accuracy on CICIDS2017 dataset with 34.51ms average inference time per flow. Real-time threat streaming via SSE demonstrates sub-100ms end-to-end latency from Python service to frontend displays. The interactive security laboratory provides hands-on educational capabilities for understanding security threats and detection mechanisms. Complete system integration across Python FastAPI, Node.js Express, MongoDB, and three client platforms demonstrates practical full-stack security operations implementation.

\subsection{Lessons Learned}

The development and evaluation of USOD provided valuable insights into multi-platform security operations development. Key lessons include: (1) the importance of modular architecture enabling independent development and integration of Python ML services with Node.js backends; (2) challenges in ML model generalization - models trained on CICIDS2017 (2017) show significantly reduced accuracy (2-19\%) on modern malware, highlighting the critical need for continuous model retraining with current threat data; (3) the value of Server-Sent Events for efficient real-time updates across multiple client platforms without WebSocket complexity; (4) benefits of demo/mock modes for system development and testing when full production capabilities (packet capture with admin rights) are not feasible during development; (5) importance of clearly distinguishing between implemented features, architectural designs for future work, and estimated/placeholder metrics in technical documentation.

\subsection{Future Work}

Several critical enhancements are needed to transition the system from educational/development to production enterprise deployment:

\textbf{Immediate Priorities:} (1) Retrain ML models on current threat datasets (CICIDS2018, modern malware traffic) to improve detection accuracy on contemporary attacks beyond the 2017 training data; (2) implement and deploy Hyperledger Fabric blockchain network with measured performance validation to provide immutable audit trails; (3) deploy Terraform/Ansible automation infrastructure with empirical measurement of deployment time and cost savings; (4) conduct formal penetration testing and comparative benchmarking against commercial SIEM solutions to validate detection capabilities and performance claims.

\textbf{ML Enhancements:} (1) Implement deep learning models (CNN/LSTM) for advanced pattern recognition; (2) add continuous learning pipeline for model retraining from production data; (3) integrate additional anomaly detection algorithms (LOF, autoencoders); (4) implement explainable AI (SHAP/LIME) for threat detection reasoning; (5) add predictive threat modeling capabilities currently documented but not implemented.

\textbf{Infrastructure Enhancements:} (1) Implement full packet capture with Scapy for production deployment with administrator privileges; (2) add GPU acceleration for ML inference to handle higher packet rates; (3) implement distributed processing with Spark for large-scale deployment; (4) conduct cloud-based load testing with auto-scaling validation; (5) implement comprehensive monitoring and alerting infrastructure.

\textbf{Platform Extensions:} IoT device support, edge computing integration, additional mobile platforms, and integration with existing enterprise security tools would extend the unified security approach to broader deployment scenarios.

\subsection{Impact and Implications}

USOD demonstrates the feasibility and benefits of unified multi-platform security operations for educational and development environments. The system's primary impact lies in: (1) demonstrating practical cross-platform development using modern frameworks (Next.js 15, React 19, Electron 38, React Native) with shared backend infrastructure; (2) validating the integration of Python ML services with Node.js backends for hybrid application/network security monitoring; (3) providing an educational platform through the interactive security laboratory for hands-on learning about attack patterns and detection mechanisms; (4) establishing architectural foundations for blockchain and cloud automation integration that can guide future enterprise implementations.

\textbf{Educational Value:} The system serves as a comprehensive educational tool demonstrating full-stack security operations development, ML-based threat detection, and multi-platform application architecture.

\textbf{Research Contributions:} The work contributes validated metrics on Random Forest and Isolation Forest performance for network intrusion detection on CICIDS2017, demonstrating both capabilities and limitations (particularly regarding dataset age and generalization to modern threats).

\textbf{Production Readiness:} While the system demonstrates core capabilities, transition to production enterprise deployment requires completion of blockchain implementation, cloud automation deployment, ML model retraining with current data, and comprehensive security testing. The current implementation is suitable for educational environments and development/testing scenarios, with clear roadmap for enterprise enhancement.

\textbf{Limitations:} Honest acknowledgment of placeholder metrics, future work items, and dataset limitations ensures realistic expectations and guides proper interpretation of system capabilities for both academic evaluation and practical deployment planning.