This paper has presented USOD (Unified Security Operations Dashboard), a comprehensive security operations platform that addresses critical challenges in modern cybersecurity through multi-platform integration, hybrid security detection, and blockchain-secured audit trails. The system demonstrates practical implementation of unified security operations across web, desktop, and mobile platforms with production-ready AI-enhanced threat detection and immutable logging.

\subsection{Summary of Contributions}

The primary contributions of this work include:

\begin{enumerate}
\item \textbf{Unified Multi-Platform Security Operations}: A cohesive framework providing consistent security operations across web (Next.js 15/React 19), desktop (Electron 38), and mobile (React Native/Expo 54) platforms with shared backend infrastructure and real-time data synchronization.

\item \textbf{Hybrid Threat Detection}: Integration of application-layer pattern-based detection (12 attack types) with network-layer ML-based detection using Random Forest (99.97\% accuracy on CICIDS2017) and Isolation Forest (87.33\% accuracy) models.

\item \textbf{AI-Enhanced Network Analysis}: Production-ready Python FastAPI service integrated with Node.js backend via webhooks and Server-Sent Events for real-time threat streaming with sub-100ms latency.

\item \textbf{Blockchain-Secured Logging}: Operational Hyperledger Fabric blockchain with 10-function chaincode providing immutable audit trails, cryptographic integrity verification, and 300 TPS throughput.

\item \textbf{Comprehensive Logging System}: 30 event types across application and network layers with dual-layer storage in MongoDB and blockchain for fast querying and tamper-proof records.

\item \textbf{Educational Security Laboratory}: Interactive security testing environment enabling hands-on attack simulation and detection for educational purposes.
\end{enumerate}

\subsection{Key Achievements}

The implementation achieves several validated technical milestones:

\textbf{Performance}: Sub-200ms average response times across all platforms, 34.51ms average ML inference time per flow, sub-100ms SSE streaming latency from threat detection to frontend display.

\textbf{Security Detection}: 99.97\% accuracy on CICIDS2017 dataset, 12 pattern-based attack types detected with automatic IP blocking, comprehensive logging of security events.

\textbf{Blockchain Integration}: Operational Hyperledger Fabric network with 4 Docker containers, 300 TPS throughput, under 100ms query response time, 10 chaincode functions for threat log management.

\textbf{Multi-Platform Deployment}: Consistent functionality across Next.js web application, Electron desktop application, and React Native mobile application with unified backend API and real-time synchronization.

\textbf{Real-Time Architecture}: Complete event-driven pipeline from AI threat detection through webhook integration to SSE streaming, enabling immediate threat notification across all connected clients.

\subsection{System Architecture}

The modular microservices architecture enables independent scaling and maintenance of components:

\textbf{Frontend Layer}: Three platform-specific applications (web, desktop, mobile) sharing common API integration patterns and consistent user experience design.

\textbf{Backend Layer}: Node.js Express 5 API server providing RESTful endpoints, SSE streaming, JWT authentication, and database integration.

\textbf{AI Layer}: Python FastAPI service providing ML-based threat detection, PCAP analysis, and real-time network monitoring capabilities.

\textbf{Data Layer}: MongoDB for primary data storage with comprehensive indexing, Hyperledger Fabric blockchain for immutable audit trails.

\subsection{Lessons Learned}

The development and evaluation of USOD provided valuable insights:

\begin{enumerate}
\item \textbf{Modular Architecture Benefits}: Separation between Python ML services and Node.js backend enables independent development, testing, and deployment of AI capabilities.

\item \textbf{Real-Time Integration}: Server-Sent Events provide efficient unidirectional streaming for security notifications without WebSocket complexity.

\item \textbf{Blockchain Operational Considerations}: Hyperledger Fabric deployment on Windows requires careful configuration of Docker networking and volume persistence.

\item \textbf{Cross-Platform Consistency}: Shared backend API and component patterns enable consistent functionality across web, desktop, and mobile platforms.

\item \textbf{ML Dataset Importance}: Model accuracy is highly dependent on training data representativeness; CICIDS2017 models may show reduced accuracy on modern attack patterns.
\end{enumerate}

\subsection{Future Work}

Several enhancements are planned for future development:

\textbf{ML Model Enhancement}: Retraining on current threat datasets (CICIDS2018, modern malware samples) to improve detection accuracy on contemporary attacks. Integration of deep learning models (CNN/LSTM) for advanced pattern recognition.

\textbf{Cloud Deployment}: Implementation of designed Terraform/Ansible automation for cloud deployment with auto-scaling, load balancing, and CI/CD pipeline integration.

\textbf{Blockchain Enhancement}: Migration from Solo to Raft consensus for improved fault tolerance, enabling TLS for production security, implementing multi-organization support for enterprise deployment.

\textbf{Advanced Features}: Predictive threat modeling using time series analysis, explainable AI (SHAP/LIME) for threat detection reasoning, continuous learning pipeline for model adaptation.

\textbf{Scalability Testing}: Comprehensive load testing with cloud infrastructure, distributed processing integration with Spark for high-volume deployments.

\subsection{Impact and Implications}

USOD demonstrates the feasibility and benefits of unified multi-platform security operations:

\textbf{Educational Value}: The interactive security laboratory and comprehensive documentation provide valuable educational resources for learning security operations, ML-based threat detection, and full-stack development.

\textbf{Research Contributions}: Validated metrics on Random Forest and Isolation Forest performance for network intrusion detection on CICIDS2017, demonstrating both capabilities and dataset limitations.

\textbf{Practical Architecture}: The modular architecture with well-defined APIs provides a reference implementation for integrating modern web technologies, ML services, and blockchain infrastructure.

\textbf{Open Implementation}: The complete implementation demonstrates practical patterns for multi-platform development, real-time streaming, and hybrid detection systems.

\subsection{Conclusion}

USOD successfully demonstrates that unified multi-platform security operations are achievable using modern web technologies combined with AI-enhanced detection and blockchain-secured logging. The system provides production-ready threat detection with validated performance metrics, operational blockchain infrastructure, and comprehensive multi-platform support.

The platform is suitable for educational environments and development scenarios, with clear architectural foundations for enterprise enhancement. The modular design enables incremental improvement through planned enhancements while maintaining current operational capabilities.
