In today's rapidly evolving cybersecurity landscape, organizations face unprecedented challenges in managing security operations across diverse platforms and environments. The proliferation of web applications, mobile devices, and desktop systems has created a complex security ecosystem where traditional, fragmented security tools fail to provide comprehensive protection. Security Operations Centers (SOCs) struggle with disconnected tools, manual processes, and the inability to maintain consistent security policies across multiple platforms.

\subsection{Problem Statement}

Current security operations face several critical challenges that hinder effective threat detection and response. First, security tools are typically platform-specific, requiring separate solutions for web, mobile, and desktop environments, leading to fragmented visibility and inconsistent security policies. Second, traditional threat detection relies heavily on signature-based approaches that fail to identify novel attack patterns and zero-day exploits. Third, security logs are often stored in centralized databases that are vulnerable to tampering, making forensic analysis unreliable. Fourth, deployment and maintenance of security infrastructure requires extensive manual configuration, leading to human errors and inconsistent security postures. Finally, the lack of real-time threat intelligence and automated response mechanisms results in delayed incident response and increased security risks.

\subsection{Motivation}

The motivation for developing a unified security operations platform stems from several critical factors. Cyber threats are becoming increasingly sophisticated, with attackers employing AI-powered techniques and multi-vector attacks that traditional security tools cannot effectively counter. Organizations operate in multi-platform environments where users access systems through web browsers, mobile applications, and desktop software, each requiring specialized security considerations. The need for real-time threat detection and response has become paramount, as the average time to detect a breach is 287 days, during which attackers can cause significant damage. Additionally, regulatory compliance requirements such as GDPR, HIPAA, and SOX mandate comprehensive audit trails and data protection measures that current fragmented solutions cannot adequately provide.

\subsection{Contributions}

This paper presents USOD (Unified Security Operations Dashboard), a comprehensive security operations platform that addresses the aforementioned challenges through several key contributions:

\begin{enumerate}
\item \textbf{Unified Multi-Platform Security Operations}: A single, cohesive platform providing consistent security operations across web (Next.js 15/React 19), desktop (Electron 38), and mobile (React Native/Expo 54) environments with shared backend infrastructure and real-time data synchronization.

\item \textbf{AI-Enhanced Network Threat Detection}: Integration of machine learning algorithms for real-time network traffic analysis, achieving 99.97\% accuracy on CICIDS2017 dataset using Random Forest classification, with Isolation Forest anomaly detection achieving 87.33\% accuracy.

\item \textbf{Blockchain-Secured Immutable Logging}: Operational Hyperledger Fabric blockchain implementation with 10-function chaincode for tamper-proof security log storage, enabling cryptographic integrity verification and providing reliable forensic capabilities.

\item \textbf{Real-Time Event Architecture}: Server-Sent Events (SSE) based streaming enabling sub-100ms latency from threat detection to dashboard display across all platforms, with webhook integration between Python AI service and Node.js backend.

\item \textbf{Comprehensive Evaluation}: Extensive evaluation demonstrating production-ready performance with sub-200ms response times across all platforms, 34.51ms average ML inference time, and 300 TPS blockchain throughput.
\end{enumerate}

\subsection{Paper Organization}

The remainder of this paper is organized as follows: Section II reviews related work in security operations platforms, multi-platform security solutions, and emerging technologies. Section III presents the overall system architecture and design principles. Section IV details the implementation of core security features and multi-platform applications. Section V describes the AI-enhanced network threat detection capabilities. Section VI explains the blockchain integration for secure logging. Section VII covers the cloud automation design specifications. Section VIII presents comprehensive evaluation results and performance analysis. Finally, Section IX concludes the paper and discusses future research directions.

