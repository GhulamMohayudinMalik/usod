\section{Conclusion}

Unlike the other audit logs that are identifying the truth, we undertook this project in response to a broken assumption in cybersecurity. That is not necessarily the case as we have demonstrated. A log may be altered, deleted or may have just been lost by the time it reaches the mind of a human analyst.

USOD is a demonstration that we are capable of doing it better. We do not need to make a decision between the intelligence of AI and the integrity of a blockchain. We can have both. Following the pipeline of a fast Random Forest classifier into an authorized registry we developed a system that identifies threats within milliseconds and permanently fixes the evidence. The blockchain overhead became manageable at 47 milliseconds, which is extremely cheap to guarantee forensic certitude.

\subsection{Contributions Recap}

We leave the community with four concrete advancements:
\begin{enumerate}
    \item \textbf{A solution to alert fatigue.} Our ATPS scoring does not simply add alerts; it puts a weight on the alerts creating a type of noise reduction, allowing the analysts to concentrate on the signal.
    \item \textbf{A safety net for the unknown.} Through supervised and unsupervised learning, we notice the attacks we are familiar with and those attack that we are unfamiliar with.
    \item \textbf{Proof of integrity.} A checking procedure that would immediately indicate whether somebody attempts to rewrite the dependent history in the database.
    \item \textbf{Real-time situational awareness.} A notification bus that literally travels through speed of attack, and deliver the news to the point of need of the analysts.
\end{enumerate}

\subsection{Where We Go From Here}

This is the version 1.0 and there is still lots of room to expand. Of particular interest to us is the possibility of \textbf{Federated Learning}, which would allow organizations to share their defense expertise without necessarily accessing sensitive information. The next one is \textbf{Explainable AI}, it is good to say to an analyst that this is an attack, but it is better to say that this is an attack because the inter-arrival interval between packets is impossibly irregular.

Another prospective which we hear concerns the possibility of an \textbf{automated response}. At this moment, USOD monitors and documents. As of the future, it will take action, and automatically generate firewall rules in order to prevent threats of high-confidence before a human has even drawn a breath.

We open-sourced USOD since we think that security is not to be assured but to be counted on. Hopefully this work will bring the industry a step nearer to that ideal.