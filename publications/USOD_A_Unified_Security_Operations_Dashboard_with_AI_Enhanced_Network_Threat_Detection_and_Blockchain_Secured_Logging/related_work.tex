\section{Related Work}

This part provides a literature review of previous work in the machine learning-based systems intrusion detection and blockchain-based audit systems. A systematic comparison of the USOD and other available solutions is presented in Table~\ref{tab:comparison}.

\subsection{Comparison with Existing Solutions}

\begin{table*}[htbp]
\caption{Comparison of USOD with Existing Security Solutions}
\label{tab:comparison}
\centering
\begin{tabular}{lccccccc}
\toprule
\textbf{System} & \textbf{ML Detection} & \textbf{Blockchain} & \textbf{Real-time} & \textbf{Multi-platform} & \textbf{Zero-day} & \textbf{Open Source} \\
\midrule
Splunk Enterprise \cite{c34} & \checkmark & $\times$ & \checkmark & \checkmark & $\times$ & $\times$ \\
IBM QRadar \cite{c35} & \checkmark & $\times$ & \checkmark & $\times$ & Partial & $\times$ \\
Snort IDS \cite{c36} & $\times$ & $\times$ & \checkmark & $\times$ & $\times$ & \checkmark \\
Suricata \cite{c37} & $\times$ & $\times$ & \checkmark & $\times$ & $\times$ & \checkmark \\
Wazuh \cite{c38} & Partial & $\times$ & \checkmark & \checkmark & $\times$ & \checkmark \\
Dorri et al. \cite{c7} & $\times$ & \checkmark & $\times$ & $\times$ & $\times$ & $\times$ \\
Nassar et al. \cite{c31} & $\times$ & \checkmark & $\times$ & $\times$ & $\times$ & $\times$ \\
\textbf{USOD (Ours)} & \checkmark & \checkmark & \checkmark & \checkmark & \checkmark & \checkmark \\
\bottomrule
\end{tabular}
\end{table*}

The SIEM systems that are available commercially (Splunk, QRadar) have advanced ML-driven detection capabilities, but do not provide any built-in blockchain support to assist with the recording of tampering. IDS tools based on signatures (Snort, Suricata) are used to detect attacks in real-time, but they fail to identify attacks because they are a zero-day attack. The previous blockchain-based solutions provide real-time detection but cannot identify zero-day attacks. Prior blockchain-based solutions \cite{c7, c31} paid attention to immutability, without including ML-based threat detection.

\subsection{Machine Learning for Intrusion Detection}

Machine learning application to network intrusion detection has changed greatly in the past twenty years. Breiman created the ensemble voting code known as the Random Forest by Breiman, which became the rival on the part of tabular flow features \cite{c3}. Farnaaz and Jabbar used the random forests proved to have better accuracy-to-training ratios than the SVMs using the KDD 99 dataset \cite{c6}.

Deep learning methods were developed in 2016, using CNNs and LSTMs as both processors of packet bytes. Nevertheless, Ferrag et al. \cite{c15} produced wide benchmarks demonstrating that on flow features engineered gradient-boosted trees (XGBoost \cite{c19}, LightGBM \cite{c39}) have a lead. This observation greatly affected our choice of model.

CICIDS2017 data \cite{c2, c13} was created to improve on the limitations of previous networks (KDD'99, NSL-KDD \cite{c14}) through the avenue of realistic bidirectional flows having both real-with-evolving types of attacks. The feature engineering conducted in this paper was informed by Ahmad et al. \cite{c40} that offered a detailed analysis of the importance of features in CICIDS2017.

\subsection{Anomaly Detection for Zero-Day Threats}

Supervised classifiers are unable to identify patterns of attacks that they have not been trained against. A survey of the anomaly detection methods conducted by Chandola et al. \cite{c25} single out Isolation Forest \cite{c4}, which proved to be especially efficient with high-dimensional spaces. The complexity of the algorithm is $O(n \log n)$ which makes it possible to detect on-line with a complexity that does not require the same amount of computation as distance-based algorithms such as LOF \cite{c41}.

Mirsky et al. \cite{c42} have recently reported Kitsune, the collection of autoencoders that detect intrusion of network traffic unsupervised. Though efficient, the computational demands are out of limits, with respect to deployment. We use Isolation Forest as a less heavy model which offers similar capabilities in terms of detecting anomalies.

\subsection{Blockchain for Security Audit Trails}

The old-fashioned systems of the log management are subjective to the post-compromise manipulation. The Execute-Order-Validate architecture, introduced with Hyperledger Fabric \cite{c5} allows permissioned use of blockchain with throughputs of more than 3,000 TPS \cite{c17}, which is way too high to achieve audit logging.

Dorri et al. \cite{c7} suggested blockchain security in cyber-physical systems, but aimed at the resource-constrained IoT devices. One of the most recent and similar applications, Nassar et al. \cite{c31} also developed on Ethereum but the gas charges of such frequent logging were prohibitive. Puiggalí and Cucurull explored distributed ledgers to understand integrity in evidence, but they did not incorporate ML-based detection to do that \cite{c43}.

The USOD stands out as unique by: (1) linking ML inference and blockchain anchoring with a single call, (2) only storing cryptographic hash representations in order to reduce the ledger size, and (3) offering cross-platform real-time notification.

\subsection{Real-Time Security Dashboards}

Commercial SIEM products provide visualization but rely on polling-based updates that introduce latency \cite{c34, c35}. Server-Sent Events (SSE) offer a lightweight alternative to WebSockets for unidirectional real-time data \cite{c44}. Our architecture leverages SSE for sub-second alert propagation to web and desktop clients, with adaptive polling for mobile platforms constrained by background execution policies \cite{c45}.

SIEM commercial products offer visualization but are based on polling that creates latency whilst updating them \cite{c34, c35}. WebSockets provide unidirectional data delivery but SSEs present a lighter version that can be used as a suitable alternative secondo le Céline, Carlo Santucci Grimmet \cite{c44}. Our architecture employs SSE to propagate alerts to web and desktop endpoints within a few seconds, and adjustable polling on mobile devices due to the background execution policy \cite{c45} limitation of mobile devices bottlenecks our architecture design efforts.
