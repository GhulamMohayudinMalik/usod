\section{Conclusion}

We undertook this project in response to a broken assumption in cybersecurity: that audit logs tell the truth. As we have demonstrated, this is not necessarily the case. A log can be altered, deleted, or simply corrupted by the time it reaches a human analyst.

USOD demonstrates that we can do better. We need not choose between the intelligence of AI and the integrity of a blockchain; we can have both. By coupling a high-accuracy XGBoost classifier with a permissioned ledger, we built a system that identifies threats within milliseconds and permanently anchors the evidence. The blockchain overhead of 47 milliseconds is a small price to pay for forensic certainty.

\subsection{Contributions Summary}

We leave the community with four concrete contributions:
\begin{enumerate}
    \item \textbf{A solution to alert fatigue.} Our ATPS scoring function does not merely add alerts; it prioritizes them, creating an effective noise filter that allows analysts to focus on the signal.
    \item \textbf{A safety net for the unknown.} By combining supervised and unsupervised learning, we detect both the attacks we recognize and those we have never seen before.
    \item \textbf{Proof of integrity.} A verification procedure that immediately detects any attempt to rewrite database history.
    \item \textbf{Real-time situational awareness.} A notification system that delivers threat intelligence to analysts within seconds of detection.
\end{enumerate}

\subsection{Future Directions}

This is version 1.0, and significant room for expansion remains. We are particularly interested in \textbf{Federated Learning}, which would allow organizations to share defensive intelligence without exposing sensitive traffic data. Another priority is \textbf{Explainable AI}: while it is useful to declare that something is an attack, it is far more valuable to explain \textit{why}, for example, that the inter-arrival time between packets is impossibly irregular.

A third avenue is \textbf{automated response}. Currently, USOD detects and documents. In future iterations, it could take action, automatically generating firewall rules to block high-confidence threats before a human has even begun to respond.

We have open-sourced USOD because we believe security should be verified, not assumed. The complete source code is available at \url{https://github.com/GhulamMohayudinMalik/usod}. We hope this work brings the industry one step closer to that ideal.